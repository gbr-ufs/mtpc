% Classe.
\documentclass[
  %% Opções da classe memoir.
  article, % Indica que é um artigo acadêmico
  11pt,    % Tamanho da fonte
  oneside, % Para impressão apenas no recto.
  a4paper, % Tamanho do papel.
  %% Opções da classe abntex2.
  %%% Títulos de letra maiúscula.
  chapter=TITLE,
  section=TITLE,
  subsection=TITLE,
  subsubsection=TITLE,
  %% Opções do pacote babel.
  english,
  brazilian
]{abntex2}
% Pacotes.
\usepackage[T1]{fontenc} % Permite fontes com mais glifos (letras).
\usepackage[utf8]{inputenc} % UTF-8.
%% Fonte bonita que é usada até hoje.
%% Veja: <https://tex.stackexchange.com/questions/147194/is-it-still-useful-to-load-the-lmodern-package>
\usepackage{lmodern}
\usepackage{amsmath, amsthm, amssymb, mathtools} % Coisas de matemática.
\usepackage{microtype} % Personalização de justificação.
\usepackage{indentfirst} % Indenta o primeiro parágrafo de cada seção.
% Configuração ABNT.
\usepackage[brazilian,hyperpageref]{backref} % Página com citações.
\usepackage[alf]{abntex2cite} % Citações ABNT.
\titulo{\uppercase{Inteligência Artificial na Universidade Federal de Sergipe: Como o Seu Uso pode Comprometer o Aprendizado em Disciplinas Introdutórias de Programação}}
\autor{
  Alexis Cristian Pertile de Oliveira Filho \and
  Bruno Danton Carneiro Silva \and
  Dávisson Cavalcante Costa \and
  Gabriel Ferreira Bernardo \and
  Gabriel Santos de Souza \and
  Wevellyn Victória da Silva Azevedo
}
\orientador{Marcos Venícius Santos}
\instituicao{
  Universidade Federal de Sergipe -- UFS}
\local{São Cristóvão}
\data{2025}
\preambulo{Artigo científico apresentado à Universidade Federal de Sergipe como requisito de avaliação parcial da disciplina de Métodos e Técnicas de Pesquisa para Computação e, por conseguinte, a obtenção dos seus créditos.}
% Configuração do PDF.
\makeatletter
\hypersetup{
  pdftitle={\@title},
  pdfauthor={\@author},
  pdfsubject={\imprimirpreambulo},
  pdfcreator={LaTeX with abnTeX2},
  pdfkeywords={universidade federal de sergipe}{federal university of sergipe}{inteligencia artificial}{artificial intelligence}{educação}{education},
  colorlinks=true,
  linkcolor=blue,
  citecolor=blue,
  filecolor=magenta,
  urlcolor=blue,
  bookmarksdepth=4
}
\makeatother
% Comandos
\newcommand{\imprimircapaelogo}{
  \begin{center}
    \includegraphics[scale=0.2]{img/logo_ufs}
  \end{center}
  \imprimircapa
}
% Espaçamento.
\setlrmarginsandblock{3cm}{3cm}{*}
\setulmarginsandblock{3cm}{3cm}{*}
\checkandfixthelayout
\setlength{\parindent}{1.3cm}
\setlength{\parskip}{0.2cm}
\SingleSpacing
% Documento.
\makeindex % Compila o índice.
\begin{document}

\selectlanguage{brazilian}
\frenchspacing % Espaçamento normal entre frases. O LaTeX usa o espaçamento arcaico de 2 espaços em vez de 1.

\imprimircapaelogo

\imprimirfolhaderosto

\tableofcontents

\newpage

\textual

\section{Introdução}

Historicamente, novas tecnologias integram a sociedade. Atualmente, ferramentas digitais são onipresentes \citeonline{cornelio_historia_2021}, com destaque para a Inteligência Artificial (IA) devido à sua multifuncionalidade e rapidez \citeonline{google_global_2025}. No meio acadêmico, o uso dessas ferramentas gera incertezas. A utilização inconsequente de LLMs traz riscos como a dependência criativa e decisória, limitando o desenvolvimento crítico dos universitários \citeonline{marrOs15Maiores2023}.

É fundamental debater se o uso da IA compromete o aprendizado de programação em cursos de Tecnologia da Informação. Estudo da Microsoft \citeonline{leeImpactGenerativeAI2025} indica que o uso já impacta negativamente a retenção de conhecimento. Embora focado em profissionais, o resultado é relevante para a educação. Pesquisa do MIT \citeonline{kosmynaYourBrainChatGPT2025} corroborou esse cenário: participantes que utilizaram IA tiveram desempenho inferior comparado aos que usaram mecanismos de busca ou apenas o cérebro. Mesmo ao alternarem para tarefas sem auxílio, o grupo que iniciou com IA manteve desempenho pior.

A popularização de LLMs como o ChatGPT no meio acadêmico torna o tema relevante. É necessário compreender o impacto do uso indiscriminado no amadurecimento cognitivo, incluindo a capacidade de aprendizado, tomada de decisões e pensamento crítico. A pesquisa visa investigar os efeitos do uso excessivo de geradores de texto em disciplinas de programação e suas consequências, como a redução da criatividade, raciocínio lógico, memória e atenção.

A importância da pesquisa abrange o contexto educacional e o mercado de trabalho, visto que estudantes dependentes de tecnologia serão os futuros profissionais. A análise questiona como tratar a dependência de IAs e garantir a formação crítica em uma realidade automatizada. A solução reside na administração do uso consciente e equilibrado da tecnologia, não em sua proibição, visando resultados educacionais significativos.

O objetivo geral da pesquisa é analisar as consequências do uso indiscriminado de IAs em disciplinas introdutórias de programação (Paradigmas Imperativo e Funcional) e propor hábitos responsáveis para que haja a integração saudável dessas tecnologias, preservando a criticidade e criatividade discente. Os objetivos específicos incluem: \begin{enumerate}
    \item Identificar a frequência de uso pelos alunos\item  Mapear os efeitos percebidos por professores na capacidade cognitiva \item    Observar os contextos de uso intenso da tecnologia em estudo\end{enumerate}

\bibliography{bibliografia}

\end{document}
